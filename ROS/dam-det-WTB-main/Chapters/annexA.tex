\section{Introduction}
\label{sec:annexIntro}

Faced with a constant growth in the demand for electricity, renewable energies are positioned as the best alternative to meet present and future needs without compromising the planet's resources. In addition, the degree of maturity of the technology has allowed a very rapid expansion due to its economic attractiveness, being today a very profitable business.

But it is not about growing at any price, the process of sustainable electrification requires ethical actions to consolidate the position of renewables in an increasingly environmentally conscious society, it must be an efficient growth.

To this end, the main metric to be assessed in the maintenance of wind farms is availability, \emph{i.e.}, having operational wind turbines  to produce when wind conditions are favorable. The inherent limitation of these technologies is determined by the variability of the natural sources that govern production: mainly the sun and the wind. The objective would therefore be to maximize electricity production while avoiding losses due to the unavailability of the installations.

Within wind turbines there are components that, due to their size, require cranes for reparing them, making any intervention on them more complex and causing long downtime (and loss of production). Among all these large components, the one that has the greatest impact on the cost of a wind turbine are the blades.

As explained in \ref{sec:motivation}, predictive maintenance allows to  to detect operational anomalies and potential defects in equipment and processes so that they can be addressed before failure occurs, helping to reduce unavailability. In the case of the blades, it is usually performed by the identification of visual damages which is currently a slow and tedious task. Let's remember that blades can reach 70 meters in length in a modern onshore installation, being also at a great height. In addition, the visual inspection of these blades requires stopping the machine during the inspection, positioning them so that they can be seen from the ground, requiring the assistance of qualified personnel for the on-site identification of the damage with its possible human error associated with the identification process.

Improving and automating the identification of blade damage would bring many benefits in different areas:
\begin{itemize}[]
    \item[\tiny$\blacksquare$]\textbf{}From an economic point of view, there are three major benefits which will be detailed in the following sections: early identification of failures, downtime minimization and manpower reduction.
    \item[\tiny$\blacksquare$] \textbf{}Environmentally, several significant positive impacts can be identified: avoiding oversizing of the installed power, components lifetime increase and major damages which may cause environmental impacts minimization.
    \item[\tiny$\blacksquare$] \textbf{}Regarding the social impact, wind farms are decentrally installed, helping the development of rural areas by increasing their population through the creation of new jobs.
\end{itemize}


\section{Description of the relevant impacts related with the project}
\label{sec:annexDescription}
This project aims to improve a process in the operation and maintenance of wind farms. Specifically, it improves the time spent on wind turbine blade inspections, as well as the accuracy of the inspections themselves, which translate in economic benefits.

Two main impacts in terms of time can be pointed: the improvement of the inspection time and the improvement of the unavailability time of the wind farm.

Firstly, the most direct economic benefit comes from saving the time needed to perform the blade inspection. This project aims to replace manual visual inspection with an automatic system. To perform the manual visual inspection, a qualified operator has to stop the wind turbine so that he can see from the ground one or two blades completely, helped by a device that allows to see at a greater distance with detail, he will have to scan the entire blade. In order to inspect all the angles of the three blades, it will be necessary to repeat this process of stopping and orienting the blades as many times as necessary, as well as moving around to visually cover the entire area of interest. The proposed solution allows with two stops of the blades to scan the entire area of interest in much less time, since it is not necessary to pay attention to details in the first visualized, in addition the work of collecting photos (less qualified personnel) is separated from the work of photo analysis (highly trained personnel to identify blade damage) and everything is digitally recorded for later review in greater detail if necessary.

Secondly, failures that do not require immediate action will be identified so that corrective intervention could be planned whether it is a blade replacement (or all three blades in case it is necessary due to a weight balancing issue) or whether it is a field or shop repair. Having the ability to plan the intervention allows to schedule it at a time when low wind conditions are guaranteed to minimize the loss of production and to ensure that there will be no safety problems in the intervention itself, it also allows to have the necessary materials for the repair, crane rental, if necessary, as well as to synchronize with the personnel required according to the type of intervention to be undertaken. All these variables will help to ensure that the corrective action is carried out in the shortest possible time, minimizing the loss of production, and maximizing the availability of the wind farm.

On the other hand, economically speaking, the next most important impact lies in identifying failures at an early stage before they evolve into catastrophic failures. This is a highly complex area because it is difficult to determine how a series of damages that have synergies with each other will evolve. The joint action of several damages on the same blade leads to unstable states that make it difficult to predict the evolution of the damage and the integrity of the blade itself. As in many areas of materials engineering with dynamic loads, the complexity is exponential and is often governed by chaotic equations (a small change in one of the damages could trigger large changes in the entire blade). There are more trivial cases where a good diagnosis and prognosis of the blade can be made. In those cases where a damage that could potentially evolve into catastrophic breakage of the blade is identified, we can consider that the incipient identification of faults will make it possible to extend the life of the components by means of repairs, at a much lower cost than a new blade.

From an environmental point of view, we can identify several significant positive impacts. Making the best use of the available resource will avoid oversizing of the installed power, as the same installation would generate more energy (by minimising production losses). This project contributes to improving the availability of wind farms, since the ideal to maximize the environmental benefits of renewable energy production would be zero energy losses. In addition, the lifetime of components would be increased through repairs before damages become catastrophic, which implies a reduction in the manufacture of new blades, reducing the pollution and waste generated by this industrial processes. Also, this major damages which could have a major impact on the environment would be minimised, such as loss of lubricants, adhesive, paint, glass/carbon fibre due to cracking, or the blade itself becoming detached and potentially damaging.

Regarding the social impact, it should be remembered that this project is part of the efficiency of wind farms, a type of installation that helps to develop the business fabric, jobs, production of materials and components in factories related to wind turbines. In addition, wind farms are decentralised since they are installed wherever there is high wind resource,  helping the development of rural areas by increasing their population through the creation of new jobs, and causing local enrichment. 

Finally, energy independence thanks to the production of electricity through an inexhaustible resource would help to avoid the volatility and fragility of energy prices in the current market model. For the small electricity consumers, a greater supply of electricity from renewable sources (which enter the electricity pool at zero price) should help to lower the selling price of energy, thus lowering the electricity bill.

\section{Detailed analysis on some of the main impacts}
\label{sec:annexDetailed}

In the above section (\ref{sec:annexDescription}), some aspects of this project which have an impact on the society, economy, or the environment have been introduced. In this section, the economy related aspects will be analysed in more detail.

Specifically, the economic aspects derived from the automation (in which this project aims to collaborate) of these preventive inspections, currently performed manually, will be analysed. 

In a current manual inspection, three main blocks of activities can be identified:

The first one consists of positioning the blades with respect to the tower (and the ground) and the blade pitch change. Usually, four sides per blade are covered: leedward side (LS), windward side (WS), leading edge (LE) and trailing edge (TE). When positioning the blades from the ground, two angles of two different blades (\emph{e.g.} LS of one of the blades and the WS of the consecutive blade) can be observed simultaneously. This positioning work will be common to both manual and automatic inspections from the ground. Depending on the expertise of the personnel it will take more or less time to position the blades.

The second block consists of scanning the blades once they have been positioned. This block is smaller in absolute terms than the previous block. Without evaluating the quality of manual observation compared to automatic observation and digitizing, it is estimated that digital scanning takes 0.5 minutes per blade section compared to an average of 3 minutes per blade section for manual scanning. This scan will be performed 12 times per wind turbine (see Figure \ref{fig:TSR_esq}).

Finally, the third block consist of the analysis of the taken images. In the manual case, the images taken in block two qualified personnel to identify possible damage. Then, they are analyzed are analyzed in more detail on a computer out of the wind farm, focusing on the images where they have found damages. In the case of automatic identification, all the photographs would be uploaded to the system that allowing to evaluate all these images of the wind turbine in a few seconds, identifying if there is damage and classifying them. To ensure good reliability, verification by qualified personnel of automatically classified damage is recommended. Another benefit of the automation of inspections is that qualified personnel are only required in the third block (while in the manual case it was necessary in the second and third blocks).

The following time savings, which are translated into manpower hours, can be estimated (Table \ref{tab:insp_dur}):

%%%%%%%%%%%%%%%%%%%%%%%%%%
\begin{table}[h!]
\begin{center}
\vskip\baselineskip
\caption{Inspection duration per wind turbine (hours).}
\label{tab:insp_dur}
\begin{tabular}{lcc}
\noalign{\smallskip}\hline\noalign{\smallskip} 
\textbf{} & \textbf{Manually} & \textbf{Automation}\\
\noalign{\smallskip}\hline\noalign{\smallskip} 
Block 1: Positioning & 1.5 (63\%)& 1.5 (88\%)\\
Block 2: Sweep & 0.4 (17\%)& 0.1 (6\%)\\
Block 3: Damage identification(*) & 0.5 (21\%)& 0.1 (6\%)\\
\noalign{\smallskip}\hline\noalign{\smallskip} 
\textbf{Total} & \textbf{2.4 (100\%)} & \textbf{1.7 (100\%)} \\
\hline
\end{tabular}}
\end{center}
\end{table}
%%%%%%%%%%%%%%%%%%%%%%%%%%

It should be noted that blade inspections are normally performed mannually in complete fleets. Considering the incessant growth of the wind farm, the possible benefits of automating blade inspections are very attractive from a pure economic point of view.


\section{Conclusions}
\label{sec:annexConclusions}

This project is part of an already environmentally responsible framework. Renewable energies are a realistic alternative for a sustainable and responsible electrification for future generations. It is a legitimate duty of this alternative to make maximum use of the energy available from renewable sources. It would not be ethically acceptable to repeat unsustainable growth models only pursued by the voracity of corporate growth. It is mandatory to minimize losses whenever possible, and within this maxim, this project fits.

In this analysis, different potential economic, social and environmental impacts of the project have been described. Environmentally, the benefits inherent in the use of renewable energy sources have been discussed, as well as the possible negative impacts it could prevent. With regard to the social aspects, the benefits which may impact on the household economy, along with the decentralised jobs generated by the installation of wind farms, helping to revitalise rural areas have been discussed.

Economically, this project aims to improve a basic process in the maintenance of wind farms. It is about merging a purely traditional type of inspection such as visual inspection with the most sophisticated image analysis technology, improving the ability to identify damage using much less time.

The results are encouraging, as the improvement in time and predictive capacity are promising. It is still difficult to evaluate the potential improvement it can bring to wind farm maintenance as some issues are difficult to quantify, more time will be needed to accumulate enough data to get an idea of the positive impact that can be obtained.

As a consequence of this improvement in inspections, it directly helps to make the transition to renewable energies even more attractive, encouraging its expansion, which generates employment, enriches the environment where wind farms are installed (usually rural environment), strengthens the industrial fabric and allows independence from other polluting energy sources exported from other markets.






